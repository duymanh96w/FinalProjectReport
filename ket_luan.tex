\chapter*{Kết luận}
Đồ án đã trình bày những nội dung lý thuyết sau:
\begin{itemize}
    \item Tổng quan về bài toán tối ưu 
    \item Các phương pháp giải bài toán tối ưu 
    \item Lý thuyết giải thuật di truyền 
    \item Các khái niệm cơ bản trong lý thuyết đồ thị, bài toán luồng cực đại trong mạng
    \item Bài toán tối ưu tuổi thọ mạng cảm biến không dây 
    \item Đề xuất mô hình bài toán tối ưu tuổi thọ mạng cảm biến không dây ngầm WUSN
    \item Đề xuất hai giải thuật giải quyết bài toán WUSN 
\end{itemize}
Về mặt thực nghiệm, đồ án đạt được các kết quả:
\begin{itemize}
    \item Cài đặt giải mô hình quy hoạch nguyên nới lỏng 
    \item Cài đặt thành công giải thuật GAH và LSMF
    \item Chạy thử nghiệm trên 20 bộ dữ liệu trích xuất từ địa hình thực tế
    \item Kiểm nghiệm sự đúng đắn của hàm mục tiêu mới bằng cách thử nghiệm thay đổi trọng số trong hàm mục tiêu 
    \item So sánh kết quả của các đề xuất với kết quả của mô hình quy hoạch nguyên nới lỏng 
\end{itemize}

Do những hạn chế về kinh nghiệm bản thân, kiến thức cũng như  thời gian, đồ án còn một số điểm yếu sau:
\begin{itemize}
    \item Đối với giải thuật GAH chưa đưa ra được phép mã hóa cũng như lai ghép, đột biến chứa nhiều đặc điểm riêng của mô hình, các cá thể vẫn mang tính khái quát cao 
    \item Đối với giải thuật LSMF, chưa tìm được cách cập nhật nhanh hàm lượng giá khi xét các hàng xóm của lời giải
\end{itemize}

Về hướng phát triển sẽ cố gắng hoàn thiện những hạn chế trên, đồng thời sử dụng các cách tiếp cận khác nhau sát với thực tế hơn cho mô hình.