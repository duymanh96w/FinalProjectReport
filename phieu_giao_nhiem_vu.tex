\begin{center}
    \textbf{PHIẾU GIAO NHIỆM VỤ ĐỒ ÁN TỐT NGHIỆP}
\end{center}

\textbf{1. Thông tin sinh viên}
\\Họ và tên: NGUYỄN DUY MẠNH 
\\Lớp: CNTT2.2 - K59 \qquad\qquad\qquad\qquad\qquad{ }{} Hệ đào tạo: Đại học chính quy 
\\Điện thoại liên lạc: 0975713207 \qquad Email: nguyenduymanhbk59@gmail.com 
\\Đồ án tốt nghiệp được thực hiện tại: bộ môn Khoa học máy tính, viện Công nghệ thông tin và truyền thông, trường Đại học Bách Khoa Hà Nội
\\Thời gian làm đồ án tốt nghiệp: từ ngày 12/01/2019 đến ngày 23/05/2019

\textbf{2. Mục đích nội dung của ĐATN}
\begin{itemize}
    \item Tìm hiểu bài toán tối ưu tuổi thọ mạng cảm biến không dây 
    \item Đề xuất giải thuật metaheuristic giải bài toán tối ưu tuổi thọ mạng cảm biến không dây 
\end{itemize}

\textbf{3. Các nhiệm vụ cụ thể của ĐATN}
\begin{itemize}
    \item Tìm hiểu bài toán tối ưu tuổi thọ mạng cảm biến không dây 
    \item Đề xuất giải thuật metaheuristic giải bài toán tối ưu tuổi thọ mạng cảm biến không dây 
    \item Tiến hành cài đặt các giải thuật đề xuất 
    \item Thực nghiệm, so sánh, đánh giá kết quả 
\end{itemize}

\textbf{4. Lời cam đoan của sinh viên}
\\Tôi – Nguyễn Duy Mạnh - cam kết ĐATN là công trình nghiên cứu của bản thân tôi dưới sự hướng dẫn của PGS.TS. Huỳnh Thị Thanh Bình và ThS. Nguyễn Thị Tâm. Các kết quả nêu trong ĐATN là trung thực, không phải là sao chép toàn văn của bất kỳ công trình nào khác.
\begin{table}[H]
    \raggedleft
    \begin{tabular}{c}
    \textit{Hà Nội, ngày 23 tháng 05  năm 2019} \\
    Tác giả ĐATN                                \\
                                                \\
    \textit{Nguyễn Duy Mạnh}                   
    \end{tabular}
\end{table}

\textbf{5. Xác nhận của giáo viên hướng dẫn về mức độ hoàn thành của ĐATN và cho phép bảo vệ}
\begin{table}[H]
    \raggedleft
    \begin{tabular}{c}
    \textit{Hà Nội, ngày\qquad tháng \qquad năm 2019} \\
    Giáo viên hướng dẫn                          \\
                                                \\
    \textit{Huỳnh Thị Thanh Bình}               
    \end{tabular}
\end{table}